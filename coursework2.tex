\documentclass{article}

\usepackage{amsmath}
\usepackage{amssymb}
\usepackage[margin=1in]{geometry}

\title{CM20256 - Coursework 2}
\date{02/05/2017}
\author{Adam Jaamour (aj645)}

% ____________________________________ DOCUMENT _____________________________________

\begin{document}

	% ________________________________ COMMANDS _________________________________

	\newcommand{\lamb}{$\lambda$}
	\newcommand{\la}{\lambda}
	\newcommand{\be}{$\beta$}
	
	\newcommand{\sa}{\quad}
	\newcommand{\saq}{\, \, \, \,}	
	
	\newcommand{\equals}{\rightarrow_\beta}
	\newcommand{\equalsT}{\sa \rightarrow_\beta^*\sa}
	
	\newcommand{\tim}{\text{times}}
	
	
	% _______________________________ PRESENTATION ________________________________
	
	\pagenumbering{gobble}
	\maketitle
	\newpage
	
	\pagenumbering{gobble}
	\begin{Large}
		\tableofcontents
	\end{Large}
	\newpage
	
	\pagenumbering{arabic}
	
	% _________________________________ PART1 __________________________________
	
	\section{Part 1 (10\%)}

	\begin{Large}
		\textbf{Show that the term \textit{``[3,2,1] times 1''} $\beta$-reduces to 6:}
		\begin{align*}
			&= \saq [3,2,1] \ \tim \ 1 \\
			&= \saq (\la c. \la n . \ c \ 3 \ (c \ 2 \ (c \ 1 \ n)) \ \tim ) \ 1 \\
			&= \saq (\la n . \ c \ 3 \ (c \ 2 \ (c \ 1 \ n)) \ [\tim \ / \ c ]) \ 1 \\
			&\equals (\la n . \ \tim \ 3 \ (\tim \ 2 \ (\tim \ 1 \ n))) \ 1 \\
			&= \saq \tim \ 3 \ (\tim \ 2 \ (\tim \ 1 \ n)) \ [1 \ / \ n] \\
			&\equals \tim \ 3 \ (\tim \ 2 \ (\tim \ 1 \ 1)) \\
			&\rightarrow \: \, \tim \ 3(\tim \ 2 \ (1 \times 1)) \\
			&= \saq \tim \ 3 \ (\tim \ 2 \ (1)) \\
			&\rightarrow \: \, \tim \ 3 \ (2 \times 1) \\
			&= \saq \tim \ 3 \ (2) \\
			&\rightarrow \: \, 3 \times 2 \\
			&= \saq 6
		\end{align*}
		``$\tim \ m \ n \ \rightarrow n \times m$'' was used for this part.
	\end{Large}
	\newpage
	
	
	% _________________________________ PART 2 _________________________________
		
	\section{Part 2 (15\%)}
	
	\begin{Large}
		\textbf{Reduce \textit{``cons 3 [2,1]''} to \textit{``[3,2,1]''}:}
		\begin{align*}
			&= \saq cons \ 3 \ [2,1]  \\
			&= \saq (((\la x.\la l.\la c.\la n \ c \ x \ (l \ c \ n))\ 3) \ [2,1] \\
			&= \saq (\la l.\la c.\la n \ c \ x \ (l \ c \ n) \ [3 \ / \ x] \ ) \ [2,1] \\
			&\equals (\la l.\la c.\la n \ c \ 3 \ (l \ c \ n)) \ [2,1] \\
			&= \saq \la c.\la n \ c \ 3 \ (l \ c \ n) \ [ \ [2,1] / l \ ] \\
			&\equals \la c.\la n \ c \ 3 \ ([2,1] \ c \ n) \\
			&= \saq \la c.\la n \ c \ 3 \ (\la c.\la n. \ c \ 2 \ (c \ 1 \ n) \ c \ n) \\
			&= \saq \la c.\la n \ c \ 3 \ (\la n. \ c \ 2 \ (c \ 1 \ n) \ [c \ / \ c] \ n) \\
			&\equals \la c.\la n \ c \ 3 \ (\la n. \ c \ 2 \ (c \ 1 \ n) \ n) \\
			&= \saq \la c.\la n \ c \ 3 \ ( c \ 2 \ (c \ 1 \ n) \ [n \ / \ n]) \\
			&\equals \la c.\la n \ c \ 3 \ ( c \ 2 \ (c \ 1 \ n)) \\
			&= \saq [3,2,1]
		\end{align*}
		``$cons \triangleq \la x.\la .l.\la x.\la n. \ c \ x \ (l \ c \ n)$'' was used for this part.
	\end{Large}
	\newpage
	
	
	% _________________________________ PART 3 _________________________________
	
	\section{Part 3 (15\%)}

	\begin{Large}
		\textbf{Define terms \textit{``head''} and \textit{``empty''} such that:}
		\begin{align*}
			head \sa [ N, ...] &\equalsT  N\\
			empty \sa [\sa] &\equalsT true\\
			head \sa [ N, ...] &\equalsT false
		\end{align*}
		\newline
		
		The \lamb -term found for \textit{``head''} is:
		\begin{equation*}
			head \triangleq \la l.l \ (\la .x.\la y.x) \ n
		\end{equation*}
		
		The \lamb -term found for \textit{``empty''} is:
		\begin{equation*}
			empty \triangleq \la l.l \ (\la .a.\la b.false) \ true
		\end{equation*}
		\newline
		
		Proof by example using the \lamb -terms found for  \textit{``head''} and \textit{``empty''}:
		
		\begin{align*}
			head \ [2,1] &\triangleq \saq (\la l.l \ (\la .x.\la y.x) \ n) \ \la c.\la n. \ c \ 2 \ (c \ 1 \ n) \\
			&= \saq (l \ (\la .x.\la y.x) \ n) \ [\la c.\la n. \ c \ 2 \ (c \ 1 \ n) \ / \ l] \\
			&\equals (\la c.\la n. \ c \ 2 \ (c \ 1 \ n)) \ (\la .x.\la y.x) \ n \\
			&= \saq (\la n. \ c \ 2 \ (c \ 1 \ n) \ [\la .x.\la y.x \ / \ c] ) \ n \\
			&\equals \la n.\ ((\la .x.\la y.x) \ 2) \ (((\la .x.\la y.x) \ 1) \ n) \ n \\
			&= \saq (((\la .x.\la y.x) \ 2) \ (((\la .x.\la y.x) \ 1) \ n)) \ [n / n] \\
			&\equals ((\la .x.\la y.x) \ 2) \ (((\la .x.\la y.x) \ 1) \ n) \\
			&= \saq ((\la .x.\la y.x) \ 2) \ (\la y.x \ [1 / x] \ n) \\
			&\equals ((\la .x.\la y.x) \ 2) \ ((\la y.1 ) \ n) \\
			&= \saq ((\la .x.\la y.x) \ 2) \ (1 \ [n/y]) \\
			&\equals ((\la .x.\la y.x) \ 2) \ 1 \\
			&= \saq (\la y.x\ [2/x]) \ 1 \\
			&\equals (\la y.2) \ 1 \\
			&= \saq 2 \ [1/y] \\
			&\equals 2
		\end{align*}
		
		\begin{align*}
			empty \ [ \ ] &\triangleq \saq (\la l.l \ (\la .a.\la b.false) \ true) \ \la c.\la n.n \\
			&= \saq (l \ (\la .a.\la b.false) \ true) \ [\la c.\la n.n \ / \ l] \\
			&\equals \la c.\la n.n \ (\la .a.\la b.false) \ true \\
			&= \saq \la n.n \ [\la .a.\la b.false \ / \ c] \ true \\
			&\equals (\la n.n) \ true \\
			&= \saq n \ [true \ / \ n] \\
			&\equals true
		\end{align*}
		
		\begin{align*}
			empty \ [1,2] &\triangleq \saq (\la l.l \ (\la .a.\la b.false) \ true) \ \la c.\la n. \ c \ 2 \ (c \ 1 \ n) \\
			&= \saq (l \ (\la .a.\la b.false) \ true) \ [\la c.\la n. \ c \ 2 \ (c \ 1 \ n) \ / \ l] \\
			&\equals (\la c.\la n. \ c \ 2 \ (c \ 1 \ n)) \ (\la .a.\la b.false) \ true \\
			&= \saq (\la n. \ c \ 2 \ (c \ 1 \ n)) \ [\la .a.\la b.false \ / \ c] \ true \\
			&\equals (\la n. \ ((\la .a.\la b.false) \ 2) \ (((\la .a.\la b.false) \ 1) \ n)) \ true \\
			&= \saq ((\la .a.\la b.false) \ 2) \ (((\la .a.\la b.false) \ 1) \ n \ [true/n]) \\
			&\equals ((\la .a.\la b.false) \ 2) \ (((\la .a.\la b.false) \ 1) \ true) \\
			&= \saq ((\la .a.\la b.false) \ 2) \ ((\la b.false \ [1/a]) \ true) \\
			&\equals ((\la .a.\la b.false) \ 2) \ ((\la b.false) \ true) \\
			&=\saq ((\la .a.\la b.false) \ 2) \ (false \ [true/b]) \\
			&\equals ((\la .a.\la b.false) \ 2) \ false \\
			&= \saq (\la b.false \ [2/a]) \ false \\
			&\equals (\la b.false) \ false \\
			&= \saq false \ [false/b] \\
			&\equals false
		\end{align*}
		\newline
		
		``$ \ [ \ ] \ = \ \la c.\la n.n$'' , where $[ \ ]$ is the empty list, was used for this part.
		
	\end{Large}
	\newpage


	% _________________________________ PART 4 _________________________________
	
	\section{Part 4 (35\%)}

	\begin{Large}
	
	\subsection{List represented by $L_m$}
	
		\textbf{What does $L_m$ = \lamb c.\lamb n.$L'_m$ (for any m) represent?}\\
		
		The list represented by the term $L_m$ corresponds to the list of descending natural numbers from $m$ to $1$, excluding 0, such as m $\in \ \mathbb{Z}^*$.
		\newline
		
		Proof by example, using $m \ = \ 4$:
		\begin{align*}
			L_4 &= \la c.\la n. \ L_4^\prime \\
			&= \la c.\la n. \ c \  4 \ ( L_3^\prime ) \\
			&= \la c.\la n. \ c \  4 \ ( c \ 3 \ (L_2^\prime )) \\
			&= \la c.\la n. \ c \  4 \ ( c \ 3 \ (c \ 2 \ (L_1^\prime ))) \\
			&= \la c.\la n. \ c \  4 \ ( c \ 3 \ (c \ 2 \ (c \ 1 \ (L_0^\prime )))) \\
			&= \la c.\la n. \ c \  4 \ ( c \ 3 \ (c \ 2 \ (c \ 1 \ n))) \\
			&= [5,4,3,2,1]
		\end{align*}
		\newline
		
	\subsection{Inductive proof}

		\textbf{Prove by induction on $m$ that $L'_m$ [ times/c , 1/n ] $\equals^*$  m!}\\
		
		\underline{Base case}: $m=0$
		\begin{align*}
			L_0^\prime \ [\ \tim / c \ , \ 1/n \ ] &= \saq n \ [\ \tim / c \ , \ 1/n \ ] \\
			&\equals 1
		\end{align*}
		
		Since $0! = 1$, the base case is therefore true.
		\newline
		
		\underline{Inductive case}: \\
		
		The inductive hypothesis is $L'_m$ [ times/c , 1/n ] $\equals^*$  m!, and it is assumed to be true.
		If it can be proved with $m+1$, then $L'_m$ [ times/c , 1/n ] $\equals^*$  m! will be true for all $m$.
		
		For $m+1$:
		\begin{align*}
			&L_{m+1}^\prime \ [\ \tim / c \ , \ 1/n \ ] \\
			&= \saq (c \ (m+1) \ L_{m-1+1}^\prime ) \ [\ \tim / c \ , \ 1/n \ ] \\
			&\equals \tim \ (m+1) \ ( L_m^\prime \ [\ \tim / c \ , \ 1/n \ ]) \\
			&= \saq \tim \ (m+1) \ m! \\
			&\rightarrow \: \, (m+1) \times  m! \\
			&= \saq (m+1)!
		\end{align*}
		
		On the second line of solution, $L_m^\prime \ [\ \tim / c \ , \ 1/n \ ]$ is substitued with $m!$ throughout the inductive hypothesis. \\
		
		``$\tim \ m \ n \ \rightarrow n \times m$'' was also used for this part. \\
		
		True for $m+1$, therefore $L'_m$ [ times/c , 1/n ] $\equals^*$ is true for all $m$. \\
		\newline
		
		\textbf{Based on previous answer, prove that $L_m$ times 1  $\equals^*$ m!}
		
		\begin{align*}
			&L_m^\prime \ \tim \ 1 \\
			&= \, (\la c.\la n. \ L_m^\prime \ \tim) \ 1 \\
			&\rightarrow \la n. \ L_m^\prime \ [\tim \ / \ c] \ 1 \\
			&\rightarrow L_m^\prime \ [\tim /c \ , \ 1/n] \\
			&= \, m!
		\end{align*}
		
		Previous proof that $L'_m$ [ times/c , 1/n ] $\equals^*$  m! used to find $m!$
	\end{Large}
	\newpage
	
	
	% _________________________________ PART 5 _________________________________
	
	\section{Part 5 (25\%)}
	
	\begin{Large}
		\textbf{Give a \lamb -term corresponding to Haskell function \textit{foldr} such as:}\\
		
		\begin{equation*}
			foldr \ f \ u \ [N_1 \ ,..., \ N_k] \equalsT f \ N_1\ (f \ N_2 \ ( \ ... \ (f \ N_k \ u)))
		\end{equation*}
		\newline
		
		answer here\\
		
		\textbf{Give a \lamb -term corresponding to Haskell function \textit{map} such as:}\\
		
		\begin{equation*}
			map \ f \ [N_1\ ,..., \ N_k] \equalsT [f \ N_1 \ , f \ N_2 \ ,...,f \ N_k]
		\end{equation*}
		\newline
		
		answer here\\
		
		\textbf{Give a \lamb -term corresponding to infinite list $[0,1,2,...]$:}\\
		
		answer here
	\end{Large}
	
	% _________________________________ TODO _________________________________
	
	\newpage
	
	\section{ToDo}
	
	\begin{Large}
		underline redexes\\
		
		check all equal, rightarrow, beta reductions and alpha reductions are correctly placed\\
		
		part 2: add alpha reduction for list\\
		
		finish part 3\\
		
		complete part 5\\
		
		replace right arrows for (times m n = m x n) with beta reductionssd 
	\end{Large}

\end{document}