\documentclass{article}

\usepackage{amsmath}
\usepackage[margin=1in]{geometry}
\usepackage{amssymb}

\title{CM20256 - Coursework 2}
\date{02/05/2017}
\author{Adam Jaamour (aj645)}

% _________________________________ DOCUMENT __________________________________

\begin{document}

	% _________________________________ SETUP __________________________________

	\newcommand{\lamb}{$\lambda$}
	\newcommand{\la}{\lambda}
	\newcommand{\be}{$\beta$}
	
	\newcommand{\sa}{\quad}
	
	\newcommand{\equals}{\rightarrow_\beta}
	\newcommand{\equalsT}{\sa \rightarrow_\beta^*\sa}
	
	\newcommand{\tim}{\text{times}}
	
	\pagenumbering{gobble}
	\maketitle
	\newpage
	
	\pagenumbering{gobble}
	\tableofcontents
	\newpage
	
	\pagenumbering{arabic}
	
	% _________________________________ PART1 __________________________________
	
	\section{Part 1 (10\%)}

	\begin{Large}
		\textbf{Show that the term \textit{``[3,2,1] times 1''} $\beta$-reduces to 6:}
		\begin{align*}
			&= [3,2,1] \ \tim \ 1 \\
			&= (\la c. \la n . \ c \ 3 \ (c \ 2 \ (c \ 1 \ n)) \ \tim ) \ 1 \\
			&= (\la n . \ c \ 3 \ (c \ 2 \ (c \ 1 \ n)) \ [\tim \ / \ c ]) \ 1 \\
			&\equals (\la n . \ \tim \ 3 \ (\tim \ 2 \ (\tim \ 1 \ n))) \ 1 \\
			&= \tim \ 3 \ (\tim \ 2 \ (\tim \ 1 \ n)) \ [1 \ / \ n] \\
			&\equals \tim \ 3 \ (\tim \ 2 \ (\tim \ 1 \ 1)) \\
			&\rightarrow \tim \ 3(\tim \ 2 \ (1 \times 1)) \\
			&= \tim \ 3 \ (\tim \ 2 \ (1)) \\
			&\rightarrow \tim \ 3 \ (2 \times 1) \\
			&= \tim \ 3 \ (2) \\
			&\rightarrow 3 \times 2 \\
			&= 6
		\end{align*}
		``$\tim \ m \ n \ \rightarrow n \times m$'' was used for this part.
	\end{Large}
	\newpage
	
	
	% _________________________________ PART 2 _________________________________
		
	\section{Part 2 (15\%)}
	
	\begin{Large}
		\textbf{Reduce \textit{``cons 3 [2,1]''} to \textit{``[3,2,1]''}:}
		\begin{align*}
			&= cons \ 3 \ [2,1]  \\
			&= (((\la x.\la l.\la c.\la n \ c \ x \ (l \ c \ n))\ 3) \ [2,1] \\
			&= (\la l.\la c.\la n \ c \ x \ (l \ c \ n) \ [3 \ / \ x] \ ) \ [2,1] \\
			&\equals (\la l.\la c.\la n \ c \ 3 \ (l \ c \ n)) \ [2,1] \\
			&= \la c.\la n \ c \ 3 \ (l \ c \ n) \ [ \ [2,1] / l \ ] \\
			&\equals \la c.\la n \ c \ 3 \ ([2,1] \ c \ n) \\
			&= \la c.\la n \ c \ 3 \ (\la c.\la n. \ c \ 2 \ (c \ 1 \ n) \ c \ n) \\
			&= \la c.\la n \ c \ 3 \ (\la n. \ c \ 2 \ (c \ 1 \ n) \ [c \ / \ c] \ n) \\
			&\equals \la c.\la n \ c \ 3 \ (\la n. \ c \ 2 \ (c \ 1 \ n) \ n) \\
			&= \la c.\la n \ c \ 3 \ ( c \ 2 \ (c \ 1 \ n) \ [n \ / \ n]) \\
			&\equals \la c.\la n \ c \ 3 \ ( c \ 2 \ (c \ 1 \ n)) \\
			&= [3,2,1]
		\end{align*}
		``$cons \triangleq \la x.\la .l.\la x.\la n. \ c \ x \ (l \ c \ n)$'' was used for this part.
	\end{Large}
	\newpage
	
	
	% _________________________________ PART 3 _________________________________
	
	\section{Part 3 (15\%)}
	\textbf{Define terms ``head'' and ``empty'' such that:}
	
	\begin{Large}
		\begin{align*}
			head \sa [ N, ...] &\equalsT  N\\
			empty \sa [\sa] &\equalsT true\\
			head \sa [ N, ...] &\equalsT false
		\end{align*}
		answer here
	\end{Large}
	\newpage


	% _________________________________ PART 4 _________________________________
	
	\section{Part 4 (35\%)}

	\begin{Large}
		\textbf{What does $L_m$ = \lamb c.\lamb n.$L'_m$ (for any m) represent?}\\
		
		answer here
		\newline
		
		\textbf{Prove by induction on $m$ that $L'_m$ [ times/c , 1/n ] $\equals^*$  m!}\\
		
		answer here
		\newline
		
		\textbf{Based on previous answer, show that $L_m$ times 1  $\equals^*$ m!}\\
		
		answer here
	\end{Large}
	\newpage
	
	
	% _________________________________ PART 5 _________________________________
	
	\section{Part 5 (25\%)}
	
	\begin{Large}
		\textbf{Give a \lamb term corresponding to Haskell function \textit{foldr} such as:}\\
		
		\begin{equation*}
			foldr \ f \ u \ [N_1 \ ,..., \ N_k] \equalsT f \ N_1\ (f \ N_2 \ ( \ ... \ (f \ N_k \ u)))
		\end{equation*}
		\newline
		
		answer here\\
		
		\textbf{Give a \lamb term corresponding to Haskell function \textit{map} such as:}\\
		
		\begin{equation*}
			map \ f \ [N_1\ ,..., \ N_k] \equalsT [f \ N_1 \ , f \ N_2 \ ,...,f \ N_k]
		\end{equation*}
		\newline
		
		answer here\\
		
		\textbf{Give a \lamb term corresponding to infinite list $[0,1,2,...]$:}\\
		
		answer here
	\end{Large}

\end{document}
